%For arXiv, uncomment this block and comment the next block.
\documentclass{amsart}
\usepackage{amsmath,amsfonts,amssymb,amsthm}
\usepackage[english]{babel}
\usepackage{graphicx}
\usepackage{url}
\usepackage[round]{natbib}
\newcommand{\forarxiv}[1]{#1}
\newcommand{\notforarxiv}[1]{}

% % For Sys Bio, uncomment this block and comment the previous block.
% \documentclass[12pt,letterpaper]{article}
% \usepackage{fixltx2e}
% \usepackage{textcomp}
% \usepackage{fullpage}
% \usepackage{amsfonts}
% \usepackage{verbatim}
% \usepackage[english]{babel}
% \usepackage{pifont}
% \usepackage{color}
% \usepackage{setspace}
% \usepackage{lscape}
% \usepackage{indentfirst}
% \usepackage[normalem]{ulem}
% \usepackage{booktabs}
% %\usepackage{nag}
% \usepackage{natbib}
% %\usepackage{bibtex}
% \usepackage{float}
% \usepackage{latexsym}
% %\usepackage{hyperref}
% \usepackage{url}
% %\usepackage{html}
% \usepackage{hyperref}
% \usepackage{epsfig}
% \usepackage{graphicx}
% \usepackage{amssymb}
% \usepackage{amsmath}
% \usepackage{bm}
% \usepackage{array}
% %\usepackage{mhchem}
% \usepackage{ifthen}
% \usepackage{caption}
% \usepackage{hyperref}
% %\usepackage{xcolor}
% \usepackage{amsthm}
% \usepackage{amstext}
% \linespread{1.66}
% \raggedright
% \setlength{\parindent}{0.5in}
% \setcounter{secnumdepth}{0}
% \pagestyle{empty}
% \renewcommand{\section}[1]{%
% \bigskip
% \begin{center}
% \begin{Large}
% \normalfont\scshape #1
% \medskip
% \end{Large}
% \end{center}}
% \renewcommand{\subsection}[1]{%
% \bigskip
% \begin{center}
% \begin{large}
% \normalfont\itshape #1
% \end{large}
% \end{center}}
% \renewcommand{\subsubsection}[1]{%
% \vspace{2ex}
% \noindent
% \textit{#1.}---}
% \renewcommand{\tableofcontents}{}
% \bibpunct{(}{)}{;}{a}{}{,}  % this is a citation format command for natbib
% \newcommand{\forarxiv}[1]{}
% \newcommand{\notforarxiv}[1]{#1}

% show keys for eqs, etc.
% \usepackage[notref,notcite]{showkeys}

% code
\newcommand{\program}{\textsf{program}}

% theorems, etc
\newtheorem{lem}{Lemma}
\newtheorem{cor}{Corollary}
\newtheorem{prop}{Proposition}
\newtheorem{thm}{Theorem}
\newtheorem{prob}{Problem}
\newtheorem{defn}{Definition}
\newtheorem{obs}{Observation}
\newtheorem{alg}{Algorithm}

\newcommand{\eat}[1]{}



\newcommand{\FIGmassTransport}{\
\begin{figure}[ht]
\begin{center}
  \forarxiv{\includegraphics[width=13cm]{mass_transport.pdf}}
\end{center}
\caption{\
  Caption goes here.
}
\label{FIGmassTransport}
\end{figure}
}
\newcommand{\refFIGmassTransport}{1}


\begin{document}

\notforarxiv{
\begin{flushright}
Version dated: \today
\end{flushright}
\bigskip
\noindent RH: PHYLOGENETICS AND THE HUMAN MICROBIOME
\bigskip
\medskip
\begin{center}

% Insert your title:
\noindent{\Large \bf Phylogenetics and the human microbiome}
\bigskip

\noindent {\normalsize \sc
Frederick A. Matsen IV$^1$}\\
\noindent {\small \it
$^1$
Program in Computational Biology, Fred Hutchinson Cancer Research Center, Seattle, WA, 91802, USA}\\
\end{center}
\medskip
\noindent{\bf Corresponding author:} Frederick A Matsen, Program in Computational Biology, Fred Hutchinson Cancer Research Center, Seattle, WA, 91802, USA; E-mail: matsen@fhcrc.org.\\
\vspace{1in}
}

\forarxiv{\
\title{Phylogenetics and the human microbiome}
\author{Frederick A. Matsen IV}
\date{\today}
\begin{abstract}
}
\notforarxiv{
\subsubsection{Abstract}
}

The human microbiome is the collection of cells that live inside and on humans.
The Human Microbiome Project (HMP) and the Metagenomics of the Human Intestinal Tract (MetaHIT) consortium have advanced human microboime research recently by doing lots of sequencing, as well as bioinformatics and statistical development.
In this paper I review the impact of phylogenetics and tree-thinking on the methods used in the analysis of this data, as well as describing current challenges for phylogenetics coming from this type of work.

\forarxiv{
\end{abstract}
\maketitle

\section{Introduction}
}

\notforarxiv{
\noindent (Keywords: human microbiome; microbial ecology; zzz)\\
\vspace{1.5in}
}

\section{Introduction to the human microbiome}

% The HMP planned to comprehensively characterize the taxonomic
% composition of the microbiome by averaging 5,000 454 FLX 16S rRNA gene
% sequences from all 300 subjects, 18 body sites, and multiple time
% points. This design, combined with more than a 1,000-fold increase in
% sequencing throughput over the course of the HMP, forced the
% consortium to develop novel tools for processing large 16S rRNA gene
% datasets, tackling issues specific to 454 sequence data quality, and
% addressing novel biological questions that were previously
% inaccessible due to limited sample sizes.

Now we will embark on our journey.

There have been hundreds of papers concerning the human microbiome. It
is not our desire to survey all of these. Rather, our idea is to
highlight areas that may be of particular interest for the
phylogenetics community by potentially having wider applications or by
being areas for future exploration.

Please note that many of the methods and issues described in this paper will be equally useful for microbial ecology in general, but by narrowing our scope we were able to keep our paper within reasonable size limits.

Antibiotics.
\cite{dethlefsen2008pervasive,dethlefsen2011incomplete,jakobsson2010short,jernberg2007long}


\section{Tree-thinking in human microbiome research}

\subsection{phylogenetics and taxonomy}

Without a doubt, the greatest impact that phylogenetics has had on microbial ecology research is on changing our view of the taxonomic relationships between microorganisms.
In the absence of sequencing, microbiologist have to rely on microscopes and staining.
Lots of little things look alike.

The most stunning example of this is the discovery of the archaea by Woese.

However, there have been

There are projects that use phylogeny to revise taxonomy, short of PhyloCode.
GreenGenes.
The reason people are attached to taxonomy is that there is a lot of information using taxonomic labels.

Much of this work has been done using the 16s ribosomal rRNA gene.

With the advent of high throughput sequencing, there has been a lot of interest concerning which variable region to use.

\subsection{Diversity estimates using phylogenetics}

Variants of phylogenetic diversity and their use.

\subsection{Community comparison using phylogenetics}

UniFrac distance and variants.

Figure: example principal coordinates plot.



\section{Phylogenetic inference as practiced by human microbiome researchers}

Very large data sets.

Some groups are not interested in accuracy per se.

Alignment.

Networks not considered.

Species delineation problem not explicitly considered. Clustering.


Chimeric sequences are a real problem, and similar to recombination.


\section{Phylogenetic challenges for human microbiome research}

\subsection{The challenge that lies ahead: functional genes and phylogenomics}

There has been a shift from interest in community dynamics to fine-scale interactions between host and immune system.


\cite{wylie2012sequence}
\cite{chen2012associating}


\section{Discussion}

What we can expect next.

\notforarxiv{
\newpage
\section{Figure Legends}
%\FIGmassTransport
\clearpage

\newpage
}

\bibliographystyle{plainnat}
\bibliography{sbreview}
\end{document}
